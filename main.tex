\documentclass{article}

% Language setting
% Replace `english' with e.g. `spanish' to change the document language
\usepackage[english]{babel}

% Set page size and margins
% Replace `letterpaper' with `a4paper' for UK/EU standard size
\usepackage[letterpaper,top=2cm,bottom=2cm,left=3cm,right=3cm,marginparwidth=1.75cm]{geometry}

% Useful packages
\usepackage{amsmath}
\usepackage{graphicx}
\usepackage[colorlinks=true, allcolors=blue]{hyperref}

\title{Rainfall and PET Analysis of Kars}
\author{\textit{Verda Kabil}}
\date{23 March 2024}

\begin{document}
\maketitle

\date{}

\section{Introduction}

Potential evaporation represents the maximum amount of water that could evaporate from a surface under idealized conditions, such as sufficient moisture availability and no limitations like water stress. Precipitation, on the other hand, is the actual amount of water that falls onto the surface. The balance between potential evaporation and precipitation determines whether an area experiences a surplus or deficit of water. Surplus occurs when precipitation exceeds potential evaporation, leading to runoff, groundwater recharge, and increased soil moisture. Deficit occurs when potential evaporation exceeds precipitation, resulting in drying of soils, reduced groundwater recharge, and potential water stress for vegetation.


\section{Materials and Methods}

\subsection{QCIS}

The shapefile downloaded from the website of the Turkish General Directorate of Maps was used to extract the minimum-maximum latitudes and longitudes of the Kars province. Through the procedures conducted in QCIS; the maximum latitude was determined as 43.74, the minimum latitude as 42.12, the maximum longitude as 41.09, and the minimum longitude as 39.92.

\subsection{CDO and Panoply}

Panoply was used to visualize the .nc files downloaded from ERA-5 websites. CDO was utilized to extract monthly and yearly PET and P values. 
To identify the driest-wettest months and years of the data, P values were compared. The driest period indicates lowest precipitation value, while the wettest period indicates precipitation value. According to this, the lowest precipitation value is 35.28 mm, which occurs in September; while the highest precipitation value is 119.02 mm, which occurs in May. Also, the lowest average precipitation value is 43.59 mm, which occurred in 1961; while the highest average precipitation value is 80.96 mm, which occurred in 1963.


\begin{table}[htbp]
  \centering
  \caption{Monthly PET and P}
    \begin{tabular}{rrr}
    \toprule
    Months & Monthly PET & Monthly P \\
    \midrule
    1     & 9.41  & 42.10 \\
    2     & 16.74 & 47.93 \\
    3     & 38.46 & 68.01 \\
    4     & 70.55 & 93.08 \\
    5     & 106.64 & 119.02 \\
    6     & 128.50 & 89.25 \\
    7     & 144.51 & 55.69 \\
    8     & 133.51 & 44.90 \\
    9     & 94.83 & 35.28 \\
    10    & 55.22 & 51.34 \\
    11    & 24.77 & 45.10 \\
    12    & 11.98 & 40.73 \\
    \bottomrule
    \end{tabular}%
  \label{Monthly PET and P Values}%
\end{table}%

\subsubsection{Codes used in CDO}

\begin{verbatim}
verda@LAPTOP-7MQIUVS3 /bin
$ cdo sellonlatbox,42.00,43.50,40.00,41.00 pet_TR.nc karss_pet.nc
cdo sellonlatbox: Processed 3 variables [0.83s 54MB]

verda@LAPTOP-7MQIUVS3 /bin
$ cdo sellonlatbox,42.00,43.50,40.00,41.00 pet_TR.nc karss_pre.nc
cdo sellonlatbox: Processed 3 variables [3.73s 54MB]

verda@LAPTOP-7MQIUVS3 /bin
$ cdo monsum karss_pre.nc monsumkars_pre.nc
cdo monsum: Processed 3 variables [1.30s 34MB]

verda@LAPTOP-7MQIUVS3 /bin
$ cdo fldmean monsumkars_pre.nc preTimeSeriesk.nc
cdo fldmean: Processed 3 variables [0.12s 16MB]

verda@LAPTOP-7MQIUVS3 /bin
$ cdo monsum karss_pet.nc monsumkars_pet.nc
cdo monsum: Processed 3 variables [1.44s 34MB]

verda@LAPTOP-7MQIUVS3 /bin
$  cdo fldmean monsumkars_pet.nc petTimeSeriesk.nc
cdo fldmean: Processed 3 variables [0.12s 16MB]

verda@LAPTOP-7MQIUVS3 /bin
$ cdo ymonsum karss_pre.nc ymonsumkars_pre.nc
cdo ymonsum: Processed 3 variables [1.61s 32MB]

verda@LAPTOP-7MQIUVS3 /bin
$ cdo ymonsum karss_pet.nc ymonsumkars_pet.nc
cdo ymonsum: Processed 3 variables [1.41s 32MB]
\end{verbatim}













\subsection{Mann-Kendall Trend Test}

Mann-Kendall Trend Test is applied with XLSTAT on Excel. When the trend is applied to the precipitation data, the results indicate Kendall's Tau as -0.273 and Sen's Slope as -2.750. They suggests a negative trend, indicating a decrease in the variable being studied over time. But for the PET data, the results indicate Kendall's Tau as 0.121 and Sen's Slope as 2.449. Based on these results, there is a weak positive trend suggested by Kendall's tau and Sen's slope.

\begin{table}[htbp]
\centering
\begin{tabular}{l|c|c}
Data & Kendall's Tau & Sen's Value \\
\hline
PET & 0.121 & 2.449 \\
P & -0.273 & -2.150 \\
\end{tabular}
\caption{\label{tab:widgets}Mann-Kendall Trend Test Results}
\end{table}

\subsection{Thornthwaite-Mather Water Balance Graph}
The Thornthwaite-Mather Water Balance Graph was created using Python. When precipitation exceeds potential evaporation over a given period, a water surplus occurs. Conversely, when potential evaporation exceeds precipitation, a water deficit occurs. 

\begin{figure}[htbp]
    \centering
    \includegraphics[width=1\linewidth]{Figure_1.png}
    \caption{Thornthwaite-Mather Water Balance Graph}
    \label{fig:mdata}
\end{figure}

\end{document}


\subsection{How to add Comments and Track Changes}

Comments can be added to your project by highlighting some text and clicking ``Add comment'' in the top right of the editor pane. To view existing comments, click on the Review menu in the toolbar above. To reply to a comment, click on the Reply button in the lower right corner of the comment. You can close the Review pane by clicking its name on the toolbar when you're done reviewing for the time being.

Track changes are available on all our \href{https://www.overleaf.com/user/subscription/plans}{premium plans}, and can be toggled on or off using the option at the top of the Review pane. Track changes allow you to keep track of every change made to the document, along with the person making the change. 

\subsection{How to add Lists}

You can make lists with automatic numbering \dots

\begin{enumerate}
\item Like this,
\item and like this.
\end{enumerate}
\dots or bullet points \dots
\begin{itemize}
\item Like this,
\item and like this.
\end{itemize}

\subsection{How to write Mathematics}

\LaTeX{} is great at typesetting mathematics. Let $X_1, X_2, \ldots, X_n$ be a sequence of independent and identically distributed random variables with $\text{E}[X_i] = \mu$ and $\text{Var}[X_i] = \sigma^2 < \infty$, and let
\[S_n = \frac{X_1 + X_2 + \cdots + X_n}{n}
      = \frac{1}{n}\sum_{i}^{n} X_i\]
denote their mean. Then as $n$ approaches infinity, the random variables $\sqrt{n}(S_n - \mu)$ converge in distribution to a normal $\mathcal{N}(0, \sigma^2)$.


\subsection{How to change the margins and paper size}

Usually the template you're using will have the page margins and paper size set correctly for that use-case. For example, if you're using a journal article template provided by the journal publisher, that template will be formatted according to their requirements. In these cases, it's best not to alter the margins directly.

If however you're using a more general template, such as this one, and would like to alter the margins, a common way to do so is via the geometry package. You can find the geometry package loaded in the preamble at the top of this example file, and if you'd like to learn more about how to adjust the settings, please visit this help article on \href{https://www.overleaf.com/learn/latex/page_size_and_margins}{page size and margins}.

\subsection{How to change the document language and spell check settings}

Overleaf supports many different languages, including multiple different languages within one document. 

To configure the document language, simply edit the option provided to the babel package in the preamble at the top of this example project. To learn more about the different options, please visit this help article on \href{https://www.overleaf.com/learn/latex/International_language_support}{international language support}.

To change the spell check language, simply open the Overleaf menu at the top left of the editor window, scroll down to the spell check setting, and adjust accordingly.

\subsection{How to add Citations and a References List}

You can simply upload a \verb|.bib| file containing your BibTeX entries, created with a tool such as JabRef. You can then cite entries from it, like this: \cite{greenwade93}. Just remember to specify a bibliography style, as well as the filename of the \verb|.bib|. You can find a \href{https://www.overleaf.com/help/97-how-to-include-a-bibliography-using-bibtex}{video tutorial here} to learn more about BibTeX.

If you have an \href{https://www.overleaf.com/user/subscription/plans}{upgraded account}, you can also import your Mendeley or Zotero library directly as a \verb|.bib| file, via the upload menu in the file-tree.

\subsection{Good luck!}

We hope you find Overleaf useful, and do take a look at our \href{https://www.overleaf.com/learn}{help library} for more tutorials and user guides! Please also let us know if you have any feedback using the Contact Us link at the bottom of the Overleaf menu --- or use the contact form at \url{https://www.overleaf.com/contact}.

\bibliographystyle{alpha}
\bibliography{sample}

\end{document}